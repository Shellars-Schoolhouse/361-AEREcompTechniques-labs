\documentclass{article}

% Adds the url command
\usepackage{hyperref}
% To remove paragraph indentation
\usepackage{parskip}
%adds url command
\usepackage{url}
\usepackage{algpseudocode}
\usepackage{algorithm}
%adds color options
\usepackage{xcolor}




\title{AERE 361: Lab 6}
\date{Due 1 MAR 2022}
\author{Sean Hellar}


\begin{document}

\maketitle
\newpage

\section{Exercise 2}

\textbf{Algorithm and Design}
\begin{itemize}
\item{My program will use if loops to ask for user input and determine if it is credible}
\end{itemize}

\textbf{Complexity}
\begin{itemize}
\item{My algorithm uses as few \textbf{if} loops as I thought possible, so the least efficient part is the \textbf{while} loop that does the actual math. With higher input the computer will take more time since it must compute a power and factorial, both non-linear functions.}
\end{itemize}



\begin{algorithm}
  \caption{Algorithm for finding the factorial}

  \begin{algorithmic}
    \Statex \Comment {Required variables:}
    \State double error input
    \State double error = 1
    \State double x
    \State double intial answer
    \State double answer
    \State int iteration

    \State \Comment{creating variables given valid arguements}
    \If {x greater than 1}
    \State x = atof(i)
    \State error input = atof(j)
    \EndIf

    \State \Comment{Begins loop to ask for user input}
    \State \textbf{else}
    \State printf user variable input statement
    \State scanf for lf

    \State \Comment{Checks that x is positive integer}
    \If {x less than or eqaul 0}
    \State printf error message needed x as positive
    \State end program
    \EndIf

    \State
    \State printf ask for relative error
    \State scanf for lf

    \State \Comment{Checks that error input is within bounds}
    \If {error input not between 0 and 1}
    \State printf error for invalid number
    \State end program
    \EndIf

    \State
    \State \textbf{end else}

    \State \Comment{This is the Mclaurin series magic}
    \While{error less than error input}
    \State iteration = iteration + 1
    \State intial answer = answer
    \State answer = answer + math formula
    \State error = fabs function * ( intial answer - answer)
    \EndWhile

    \State
    \State printf final output
 
  \end{algorithmic}
\end{algorithm}

    
    








\end{document}

