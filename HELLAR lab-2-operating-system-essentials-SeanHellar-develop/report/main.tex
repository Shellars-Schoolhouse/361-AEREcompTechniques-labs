\documentclass{article}

% Adds the url command
\usepackage{hyperref}
% To remove paragraph indentation
\usepackage{parskip}

% Update these:
\title{Pure Latex Poetry}
\date{25 JAN 2022}
\author{Sean Hellar}

\begin{document}

\maketitle
\newpage

\section{Exercise 1}
Using the \texttt{ls -ls} to list the files in order of time I found that main.log was the last changed file based on the timestamp. 

Command: \texttt{ls -lt}

Disphering output:\\
From the output I looked at column with the timestamps becase it shows exactly when the files were last edited.

Options used:
\begin{itemize}
    \item{\texttt{-lt} orders the files by modification time}
\end{itemize}

Sources of informtion:
\begin{itemize}
    \item{\texttt{man ls}: found list of options in manual}
    \item{\texttt{https://www.tecmint.com/sort-ls-output-by-last-modified-date}}
\end{itemize}


\newpage
\section{Exercise 2}
I tried the wc command by itself but I didn't specify which files to look for so it didn't return anything. As instructed I needed to create a pipeline from the list of files to the wc search. 

Command: \texttt{ls | wc -l}

Disphering output:\\
My output of 4 means that there are 4 files in my report directory at this current time. 

Options used:
\begin{itemize}
    \item{\texttt{-l} prints the newline counts }
\end{itemize}

Sources of informtion:
\begin{itemize}
    \item{\texttt{man wc}: shows list of options in wc manual}
\end{itemize}

\newpage
\section{Exercise 3}
Using the \texttt{whereis ls} I was able to find the binary path of \texttt{ls} which happens to be rather short. 

Command: \texttt{whereis ls}

Disphering output:\\
There were two printed paths from this command, but as noted in the lab manual the correct path must end \texttt{ls}. 

Options used:
\begin{itemize}
    \item{\texttt{whereis ls} is just a simple mix of two commands}
\end{itemize}

Sources of informtion:
\begin{itemize}
    \item{\texttt{man whereis}: shows full description of \texttt{whereis} uses}
\end{itemize}

\newpage
\section{Exercise 4}
I used the find function and there a countless readme.txt files that I dont have permission for. I filtered out all the permission denied files so I could easily see the correct full paths.

Command: \texttt{find.-namereadme.txt2>1|grep-v"Permission denied"}
\begin{itemize}
    \item{Note: there should be an ampersand between the \texttt{>} and \texttt{one}, but the script would not save or recognize it}
\end{itemize}

Disphering output:\\
The given output displayed all files from the root directory down with name ``readme.txt''

Options used:
\begin{itemize}
    \item{\texttt{-v message}: specifically finds the file with the given error message }
\end{itemize}

Sources of information:
\begin{itemize}
    \item{\texttt{https://www.cyberciti.biz/faq/bash-find-exclude-all/}}
\end{itemize}

\newpage
\section{Exercise 5}
In a fairly simple process I used \texttt{top} to output a list of running programs, CPU and memory usage and it continued to update live. 

Command: \texttt{top}

Disphering output:\\
The numerous columns output the CPU and memory usage and at the top left corner of the output shows when it last updated.  

Options used:
\begin{itemize}
    \item{\texttt{top} did not need any additional arguments }
\end{itemize}

Sources of information:
\begin{itemize}
    \item{\texttt{https://www.cyberciti.biz/faq/how-to-check-running}}
\end{itemize}


\end{document}
